\selectlanguage{frenchb}
\section*{R\'esum\'e}
Les systèmes logiciels tendent à se doter de facultés d’adaptation, d’évolution et d’ouverture. Ces capacités requièrent une grande flexibilité et dynamicité de l’environnement d’exécution, ainsi que de nouveaux outils d’assistance à la fabrication de ces systèmes. En électronique, des outils ont été déployés pour faire face à l’hétérogénéité et au nombre de composants, ainsi qu’aux besoins d’adaptation de produits existants à de nouvelles technologies. L’ouverture de la documentation et des spécifications a permis une grande richesse de solutions venant tant de bricoleurs que d’industriels.
Inspiré par l’électronique, cette thèse contribue à l’amélioration de la flexibilité des systèmes logiciels tout en conservant un haut niveau de fiabilité. Les apports se font à trois niveaux.\\
(1) Un nouveau modèle de composants qui offre une grande flexibilité et permet la connection de composants hétérogènes. \\
(2) Des outils issus de l’ingénierie des modèles, pour créer, modifier, simuler et valider la structure et le comportement des assemblages de composants avant leur déploiement. \\
(3) Un environnement d’exécution bati sur une architecture à base de services, pour supporter les évolutions, les adaptations et l’ouverture requises par le modèle de composant proposé.\\
Cette thèse a été validée sur un cas concret dans un projet d’aide à domicile. Dans ce domaine, les systèmes logiciels doivent être adaptables et flexibles, pour répondre aux évolutions des besoins et pathologies des personnes âgées. Les bénéfices acquis de l’utilisation de cette approche dans ce contexte ont prouvé la pertinence de cette thèse.



\selectlanguage{english}
\section*{Abstract}

Software systems tend to acquire capabilities of adaptation, evolution and openness. These abilities require the execution environment to be highly flexible and dynamic, and require new tools to handle these abilities. In electronics, tools have been set up to cope with the huge heterogeneity and number of components, and the adaptation of existing products to new technologies. Openness of documentations and specifications in this area led to a wealth of solutions made by industrials or individuals fond of electronics.\\
Inspired by electronics' achievements this thesis contributes in improving the flexibility of software systems while maintaining a high level of reliability. The contribution is threefold. (1) A new component model, which improves flexibility to enable connection of heterogeneous components. (2) Tools from model driven engineering, to create, edit, simulate and validate the structure and behavior of component assemblies prior to their (re-)deployment. (3) A runtime environment built on top of a service-based architecture to support evolutions, adaptations and openness required by the proposed component model.\\
This thesis has been validated on a use case from an Ambient Assisted Living project. In this domain, software systems have to be adaptive and flexible, to fit the needs and pathology evolutions of elderly people. Although there still is a long way to go, the benefits gained from the use of this approach in this context proved the relevance of this thesis.